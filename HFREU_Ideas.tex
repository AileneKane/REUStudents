\documentclass[11pt]{article}

\usepackage[left=1in,right=1in]{geometry}                
\usepackage{graphicx}
\usepackage{amssymb}
\usepackage{amsbsy}
\usepackage{amsmath}
\usepackage{multirow}
\usepackage{lineno}
\usepackage{caption}
\usepackage{longtable}
\usepackage{setspace}
\usepackage{fancyhdr}
\usepackage{natbib}
\usepackage{subfigure}
\usepackage{booktabs}
\usepackage{lscape}

\bibliographystyle{/Users/aileneettinger/Dropbox/Library/bstfiles/pnas}
%%%%%%%%%%%%%%%%%%%%%%%%%%%%%%%%%%%%%%%%
\begin{document}
\noindent \emph{\LARGE Research Project Ideas for Harvard Forest REUs}
%%%%%%%%%%%%%%%%%%%%%%%%%%%%%%%%%%%%%%%%

\section{Effects of Urbanization on Tree Growth and Climate Sensitivity}
\subsection{Question: How does urbanization affect tree growth rates and growth sensitivity to temperature and other climatic factors?}
\subsection{Approach:}
\paragraph {Focal species:} Choose 5-10 tree species that occur at both Harvard Forest and Bussey Brook Meadows and other unmanaged urban forests around Arnold Arboretum. These species could include: \emph{Betula allegheniensis, Betula papyrfera, Fagus grandifolia, Quercus alba, Quercus rubra, Pinus strobus, Populus deltoides, Tsuga canadensis}. Add nonnative tree species, too?
\paragraph {Field work at Harvard Forest:} Core trees. Could collect additional data, such as soil nitrogen and other soil characteristics. 
\paragraph {Lab work (could be at Harvard Forest or Arnold Arboretum):} Mount, sand, and analyze cores. Gather climate data from Harvard Forest. Correlate growth to climate data (temperature precipitation) and compare growth rates and climate sensitivity to those of trees at/near Arnold Arboretum (many have already been collected, but some additional cores may need to be collected in 
Boston, depending on the focal species selected).
\paragraph {References:} Need to add a few relevant ones!

\section{}
\subsection{}
\paragraph {}

%%%%%%%%%%%%%%%%%%%%%%%%%%%%%%%%%%%%%%%%
\end{document}
%%%%%%%%%%%%%%%%%%%%%%%%%%%%%%%%%%%%%%%%
