\documentclass[11pt]{article}

\usepackage[left=1in,right=1in]{geometry}                
\usepackage{graphicx}
\usepackage{amssymb}
\usepackage{amsbsy}
\usepackage{amsmath}
\usepackage{multirow}
\usepackage{lineno}
\usepackage{caption}
\usepackage{longtable}
\usepackage{setspace}
\usepackage{fancyhdr}
\usepackage{natbib}
\usepackage{subfigure}
\usepackage{booktabs}
\usepackage{lscape}
\usepackage{/Library/Frameworks/R.framework/Resources/share/texmf/Sweave}



%%%%%%%%%%%%%%%%%%%%%%%%%%%%%%%%%%%%%%%%
\begin{document}
\noindent \emph{\LARGE Research Project Ideas for Harvard Forest REUs}
%%%%%%%%%%%%%%%%%%%%%%%%%%%%%%%%%%%%%%%%

\section{Effects of Urbanization on Tree Growth and Climate Sensitivity}
\subsection{Question: How does urbanization affect tree growth rates and growth sensitivity to temperature and other climatic factors?}
\subsection{Approach:}
\paragraph {Focal species:} Choose 5-10 tree species that occur at both Harvard Forest and Bussey Brook Meadows and other unmanaged urban forests around Arnold Arboretum. These species could include: \emph{Betula allegheniensis, Betula papyrfera, Fagus grandifolia, Quercus alba, Quercus rubra, Pinus strobus, Populus deltoides, Tsuga canadensis}. Add nonnative tree species, too?
\paragraph {Field work at Harvard Forest:} Core trees. Could collect additional data, such as soil nitrogen and other soil characteristics. 
\paragraph {Lab work (could be at Harvard Forest or Arnold Arboretum):} Mount, sand, and analyze cores. Gather climate data from Harvard Forest. Correlate growth to climate data (temperature precipitation) and compare growth rates and climate sensitivity to those of trees at/near Arnold Arboretum (many have already been collected, but some additional cores may need to be collected in 
Boston, depending on the focal species selected).
% <<<<<<< HEAD

\paragraph {References:} Need to add a few relevant ones, like O'Brian et al 2012.


\section{Functional trait diversity over agricultural intensification gradient}

\subsection{Questions: What level of agricultural intensity is sustainable for the New England landscape? Can plant functional traits reveal the signature of sustainable management to support ecosystem services?}

\paragraph{How does grazing alter the relationship between biodiversity and provisioning of ecosystem services in the New England landscape?} Agricultural land is a vital but vanishing part of the New England landscape. As forest cover expands in response to agricultural abandonment, the ecosystem services provided by these communities become increasingly valuable.

\subsection{Approach:}

\paragraph{Harvard Farm} The Harvard Farm is a recent addition to the Harvard Forest land, and presents a unique opportunity to work at the interface between conservation biology, sustainable agriculture, and basic research in ecology. The grasslands at Harvard Farm are now being managed with a range of grazing intensity by local dairy farmers, with intensive grazing, rotational grazing, and hay-making as the three intensities. Starting in 2014, vegetation and bird life have been censused to establish baseline data, providing a useful background for ongoing research.

\paragraph{Community censuses and functional trait measurements} 

\section{Tree functional traits across climate gradients}

\subsection{Question: Do populations of trees exhibit adaptive variation in functional traits across a climate gradient?}

\paragraph{Woody plants rely on ability to persist across wide variations in climate to germinate, survive, and reproduce.} Ability to maintain performance across a range of environments is hypothesized to be especially useful for early successional and invasive species, whereas dominant late successional species are assumed to have narrower functional trait range. However, testing this and related question remains a challenge at the scale of forest ecosystems.

\subsection{Approach:}

\paragraph{We will sample leaf and stem morphological and physiological characteristics for up to 40 species of woody plants at Harvard Forest.} For key species, we will sample populations across the climate ranges of these species, ranging from Connecticut to northern New Hampshire. Specific wood density, leaf toughness, specific leaf area, leaf nitrogen and phosphrous concentrations, and other traits will be compiled. Analysis of functional traits at the population and community level will test how dominance and trait variability range with climate.

% Adding some R code with Sweave. Starting R code with "<<>>=" and end with "@".

\begin{Schunk}
\begin{Sinput}
> plot(1:10)
> 
\end{Sinput}
\end{Schunk}


%%%%%%%%%%%%%%%%%%%%%%%%%%%%%%%%%%%%%%%%
\end{document}
%%%%%%%%%%%%%%%%%%%%%%%%%%%%%%%%%%%%%%%%
